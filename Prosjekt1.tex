\documentclass{article}

\usepackage[utf8]{inputenc}
\usepackage{graphicx} %package to manage images
\graphicspath{ {/Users/livewj/Documents/UiO/FYS2150/Bolgeoptikk} }


\usepackage{listings}
\usepackage{placeins}
\usepackage{subcaption}


\usepackage[rightcaption]{sidecap}
\usepackage{wrapfig}
\usepackage{color}

\definecolor{mygreen}{rgb}{0,0.6,0}
\definecolor{mygray}{rgb}{0.5,0.5,0.5}
\definecolor{mymauve}{rgb}{0.58,0,0.82}

\lstset{ %
  backgroundcolor=\color{white},   % choose the background color; you must add \usepackage{color} or \usepackage{xcolor}
  basicstyle=\footnotesize,        % the size of the fonts that are used for the code
  breakatwhitespace=false,         % sets if automatic breaks should only happen at whitespace
  breaklines=true,                 % sets automatic line breaking
  captionpos=b,                    % sets the caption-position to bottom
  commentstyle=\color{mygreen},    % comment style
  deletekeywords={...},            % if you want to delete keywords from the given language
  escapeinside={\%*}{*)},          % if you want to add LaTeX within your code
  extendedchars=true,              % lets you use non-ASCII characters; for 8-bits encodings only, does not work with UTF-8
  frame=single,                    % adds a frame around the code
  keepspaces=true,                 % keeps spaces in text, useful for keeping indentation of code (possibly needs columns=flexible)
  keywordstyle=\color{blue},       % keyword style
  language=Octave,                 % the language of the code
  morekeywords={*,...},            % if you want to add more keywords to the set
  numbers=left,                    % where to put the line-numbers; possible values are (none, left, right)
  numbersep=5pt,                   % how far the line-numbers are from the code
  numberstyle=\tiny\color{mygray}, % the style that is used for the line-numbers
  rulecolor=\color{black},         % if not set, the frame-color may be changed on line-breaks within not-black text (e.g. comments (green here))
  showspaces=false,                % show spaces everywhere adding particular underscores; it overrides 'showstringspaces'
  showstringspaces=false,          
  showtabs=false,                 
  stepnumber=2,                    
  stringstyle=\color{mymauve},     
  tabsize=2,                      
  title=\lstname                   
}


\title{Prosjekt 1}
\author{Live Wang Jensen}
\date{\today}

\begin{document}
\maketitle


\begin{abstract}
I dette prosjektet skal vi se nærmere på en numerisk løsning av den velkjente Poisson-ligningen, hvor Dirichlets grensebetingelser er brukt. Den andrederiverte er blitt tilnærmet med en tre-punkts-formel, og selve problemet løses ved hjelp av et lineært ligningssett. Vi skal løse disse ligningene på to ulike måter; ved Gauss-eliminasjon og ved LU-faktorisering. De to løsningsmetodene skal så sammenlignes når det kommer til FLOPS (floating point operations per second) og beregningstid.

\end{abstract}

\section{Introduksjon}
Vi starter med å se på den én-dimensjonale Poisson ligningen for en sfærisk kule funksjon (?). Denne ligningen løses numerisk ved å dele opp, eller disktretisere, intervallet $x$ $\epsilon$ [0,1], og løsningen $f$ på ligningen er gitt. Teoridelen viser at det å finne en diskret løsning vil være det samme som å løse et lineært ligningssett, $A\textbf{u}$ = $\textbf{\~b}$. Det vil vise seg at matrisen $A$ er en såkalt tridiagonal matrise, noe som forenkler Gauss-eliminasjonen betraktelig. Selve metoden er implementert i et Python-program, hvor vi har variert antall grid-points $n$. Den numeriske løsningen sammenlignes så med den analytiske løsningen. Deretter beregnes den maksimale relative feilen i den numeriske løsningen. Resultatet plottes som en funksjon av $n$. Helt til slutt skal vi bruke LU-faktorisering på matrisen $A$ til å finne den numeriske løsningen på Poisson ligningen. Antall FLOPS og beregningstid sammenlignes så mellom de to løsningmetodene.

\section{Teori}
Mange differensialligninger innenfor fysikk kan skrives på formen
\begin{equation}
\frac{d^2y}{dx^2} + k^2(x)y = f(x)
\end{equation}
hvor $f$ er det uhomogene leddet i ligningen, og $k^2$ er en reell funksjon. Et typisk eksempel på en slik ligning finner vi i elektromagnetismen, nemlig Poisson-ligningen

\begin{equation}
\triangle^2 \Phi = -4\pi\rho(\textbf{r})
\end{equation}

Her er $\Phi$ det elektrostatiske potensialet, mens$\rho (\textbf{r})$ er ladningsfordelingen.
Dersom vi antar at $\Phi$ og $\rho(\textbf{r})$ er sfærisk symmetriske, kan vi forenkle ligning (2) til en én-dimensjonal ligning,
\begin{equation}
\frac{1}{r^2}\frac{d}{dr}\left(r^2\frac{d\Phi}{dr}\right) = -4\pi\rho(r)
\end{equation}

Hvis vi bruker at  $\Phi(r) = \phi(r)/r$, kan ligningen skrives som
\begin{equation}
\frac{d^2\phi}{dr^2} = -4\pi r\rho(r)
\end{equation}

Det uhomogene leddet $f$ er nå gitt ved produktet $-4\pi\rho$. Dersom vi lar $\phi \rightarrow u$ og $r \rightarrow x$, ender vi opp med en generell, én-dimensjonal Poisson ligning på formen 
\begin{equation}
-u''(x) = f(x)
\end{equation}

Det er denne ligningen vi skal se nærmere på i denne oppgaven. Nærmere bestemt skal vi løse ligningen
\begin{equation}
-u''(x) = f(x),\quad  x \in (0,1), \quad    u(0) = u(1) = 0
\end{equation}

Vi skal altså holde oss innenfor intervallet $x \in (0,1)$, med Dirichlet-grensebetingelsene gitt ved $u(0) = u(1) = 0$. Vi definerer den diskrete tilnærmingen til løsningen $u$ med $v_i$, slik at $x_i = ih$, hvor $h = 1/(n+1)$ er steglengden. Vi får da at $x_0 = 0$ og $x_{n+1} = 1$. Den andrederiverte av $u$ kan da tilnærmes med en tre-punkts formel

\begin{equation}
-\frac{v_{i+1} + v_{i-1} - 2v_i}{h^2} = f_i \quad \textrm{når} \quad i = 1,...,n
\end{equation}
hvor $f_i = f(x_i)$ og grensebetingelsene er gitt som $v_0 = v_{n+1} = 0$. Feilleddet her går som $O(h^2)$.

Dersom vi multipliserer leddet $h^2$ i ligning (7) på hver side, kan vi definere leddet på høyre side av ligningen som $\tilde{b_i} = h^2f_i$. Vi ender altså opp med 

\begin{equation}
- v_{i+1} - v_{i-1} + 2v_i = \tilde{b_i}
\end{equation}

Dersom vi setter inn ulike verdier av $i$ i ligningen ovenfor, ender vi opp med en tilhørende ligning på samme form:
\[i=1: \quad v_2 + v_0 - 2v_1 = \tilde{b_1}\]
\[i=2: \quad v_3 + v_1 -2v_2 = \tilde{b_2}\]
osv.
Vi ender altså opp med et lineært ligningssett som kan settes opp som en matriseligning:

\begin{equation}
    \left(\begin{array}{cccccc}
                           2& -1& 0 &\dots   & \dots &0 \\
                           -1 & 2 & -1 &0 &\dots &\dots \\
                           0&-1 &2 & -1 & 0 & \dots \\
                           & \dots   & \dots &\dots   &\dots & \dots \\
                           0&\dots   &  &-1 &2& -1 \\
                           0&\dots    &  & 0  &-1 & 2 \\
                      \end{array} \right)\left(\begin{array}{c}
                           v_1\\
                           v_2\\
                           \dots \\
                          \dots  \\
                          \dots \\
                           v_n\\
                      \end{array} \right)
  =\left(\begin{array}{c}
                           \tilde{b}_1\\
                           \tilde{b}_2\\
                           \dots \\
                           \dots \\
                          \dots \\
                           \tilde{b}_n\\
                      \end{array} \right)
\end{equation}

hvor $A$ er en tridiagonal $nxn$-matrise. Hvis vi kaller vektorene
\begin{equation}
\left(\begin{array}{c}
                           v_1\\
                           v_2\\
                           \dots \\
                          \dots  \\
                          \dots \\
                           v_n\\
                      \end{array} \right) = \textbf{v} \quad \textrm{og} \quad 
                      \left(\begin{array}{c}
                           \tilde{b}_1\\
                           \tilde{b}_2\\
                           \dots \\
                           \dots \\
                          \dots \\
                           \tilde{b}_n\\
                      \end{array} \right) =\tilde{\textbf{b}}
\end{equation}
kan vi på forkortet form skrive ligningen som
\begin{equation}
\textbf{Av} = \tilde{\textbf{b}}
\end{equation}

Her er $A$ og $\tilde{\textbf{b}}$ kjent, mens vektoren $\textbf{v}$ er den ukjente.

I vårt tilfelle er funksjonen $f$ er gitt som $f(x) = 100e^{-10x}$. Dersom vi bruker dette i ligning (5) og integrerer denne ligningen analytisk, ender vi opp med en løsning på formen
\begin{equation}
u(x) = 1- (1-e^{-10})x-e^{-10x}
\end{equation}

Det er denne analytiske løsningen vi skal sammenligne den numeriske løsningen med. 

\section{Løsningsmetoder}
\subsection{Gauss-eliminasjon}
\textbf{Generell tridiagonal matrise}\\
Vi kan tenke oss at elementene langs diagonalen i matrisen vår, $A$, utgjør en vektor $b$, samtidig som elementene på hver side av diagonalen utgjør vektorene $a$ og $c$. Alle andre elementer i matrisen er null. Vi antar nå at elementene i hver vektor $a$, $b$ og $c$ \textit{ikke} er identiske. Matriseligningen kan da skrives

\begin{equation}
    \left(\begin{array}{cccccc}
                           b_1& c_1 & 0 &\dots   & \dots &\dots \\
                           a_2 & b_2 & c_2 &\dots &\dots &\dots \\
                           & a_3 & b_3 & c_3 & \dots & \dots \\
                           & \dots   & \dots &\dots   &\dots & \dots \\
                           &   &  &a_{n-2}  &b_{n-1}& c_{n-1} \\
                           &    &  &   &a_n & b_n \\
                      \end{array} \right)\left(\begin{array}{c}
                           v_1\\
                           v_2\\
                           \dots \\
                          \dots  \\
                          \dots \\
                           v_n\\
                      \end{array} \right)
  =\left(\begin{array}{c}
                           \tilde{b}_1\\
                           \tilde{b}_2\\
                           \dots \\
                           \dots \\
                          \dots \\
                           \tilde{b}_n\\
                      \end{array} \right).
\end{equation}

hvor vektorene $a$, $b$ og $c$ har lengden $n$. Hvis vi nå tenker oss at n=4, forenkler denne matriseligningen seg noe:
\begin{equation}
    \left(\begin{array}{cccccc}
                           b_1& c_1 & 0 &0 \\
                           a_2 & b_2 & c_2 &0 \\
                           0 & a_3 & b_3 & c_3 \\
                           0 & 0 & a_4 &  b_4  \\
                      \end{array} \right)\left(\begin{array}{c}
                           v_1\\
                           v_2\\
                           v_3\\
                          v_4  \\
                      \end{array} \right)
  =\left(\begin{array}{c}
                           \tilde{b}_1\\
                           \tilde{b}_2\\
                        \tilde{b}_3\\
                        \tilde{b}_4\\
                      \end{array} \right).
\end{equation}

Vi vil da få ligninger på formen
\[i=1: \quad b_1v_1 + c_1v_2 = \tilde{b}_1\]
\[i=2: \quad a_2v_1 + b_2v_2 + c_2v_3 = \tilde{b}_2\]
\[i=3: \quad a_3v_2 + b_3v_3 + c_3v_4 = \tilde{b}_3\]
\[i=4: \quad a_4v_3 + b_4v_4 = \tilde{b}_4\]
\[\vdots\]
\begin{equation}
a_iv_{i-1} + b_iv_i + c_iv_{i+1} = \tilde{b}_i \quad \textrm{når} \quad i = 1,2,...,n
\end{equation}

For å kunne løse dette ligningssettet, bruker vi metoden \textit{forlengs substitusjon}. Det vi ønsker er å få element $a_1$ til å bli null. Vi starter med å multiplisere første rad i matrisen med $a_2/b_1$. Deretter trekker vi første rad fra andre rad. Skriver vi matrisen på utvidet form vil vi få:

\[
\left[
\begin{array}{cccc|c}
1 & c_1/b_1 & 0 & 0 & \tilde{b}_1/b_1 \\
 0 & b_2 - \frac{c_1}{b_1}a_2  &c_2 &0 & \tilde{b_2} - \frac{\tilde{b_1}}{b_1}a_2\\
  0 & a_3 & b_3 & c_3 & \tilde{b_3}\\
  0 & 0 & a_4 & b_4 & \tilde{b_4}\\
\end{array}
\right]
\]

Nå som første element i andre rad har blitt redusert til 0, kan vi gå i gang med å redusere andre element i tredje rad. La oss først, for ordens skyld, definere $\tilde{d_1} = b_1$, $\tilde{d_2} = b_2 - \frac{c_1}{b_1}a_2$, $\tilde{e_1} = \tilde{b_1}/\tilde{d_1}$ og $\tilde{e_2} =  (\tilde{b_2} - \frac{\tilde{b_1}}{b_1}a_2)/\tilde{d_2}$. Vi kan nå multiplisere rad nummer to med $1/\tilde{d_2}$ og døpe om variablene til noe mer oversiktelig: 

\[
\left[
\begin{array}{cccc|c}
1 & c_1/\tilde{d_1} & 0 & 0 & \tilde{b}_1/\tilde{d_1} \\
 0 & 1  &c_2/\tilde{d_2} &0 & (\tilde{b_2} - \frac{\tilde{b_1}}{b_1}a_2)/\tilde{d_2}\\
  0 & a_3 & b_3 & c_3 & \tilde{b_3}\\
  0 & 0 & a_4 & b_4 & \tilde{b_4}\\
\end{array}
\right] \sim
\left[
\begin{array}{cccc|c}
1 & c_1/b_1 & 0 & 0 & \tilde{e_1} \\
 0 & 1  &c_2/\tilde{d_2} &0 &\tilde{e_2}\\
  0 & a_3 & b_3 & c_3 & \tilde{b_3}\\
  0 & 0 & a_4 & b_4 & \tilde{b_4}\\
\end{array}
\right]
\]

Trekker nå $a_3 \cdot$(rad to) fra rad tre:
\[
\left[
\begin{array}{cccc|c}
1 & c_1/\tilde{d_1} & 0 & 0 & \tilde{e_1} \\
 0 & 1  &c_2/\tilde{d_2} &0 &\tilde{e_2}\\
  0 & 0 & b_3 - \frac{c_2}{\tilde{d_2}}a_3 & c_3 & \tilde{b_3} - \tilde{e_2}a_3\\
  0 & 0 & a_4 & b_4 & \tilde{b_4}\\
\end{array}
\right] \sim
\left[
\begin{array}{cccc|c}
1 & c_1/\tilde{d_1} & 0 & 0 & \tilde{e_1} \\
 0 & 1  &c_2/\tilde{d_2} &0 &\tilde{e_2}\\
  0 & 0 & 1 & c_3/\tilde{d_3} &  (\tilde{b_3} - \tilde{e_2}a_3)/\tilde{d_3}\\
  0 & 0 & a_4 & b_4 & \tilde{b_4}\\
\end{array}
\right]
\]

I den siste overgangen har vi multiplisert tredje rad med $1/\tilde{d_3}$. Vi ser et gjentakende mønster hvor 
\begin{equation}
\tilde{d_i}= b_i - \frac{c_{i-1}}{\tilde{d_{i-1}}}a_i \quad \textrm{og} \quad \tilde{e_i}=\frac{\tilde{b_i}-\tilde{e_{i-1}a_i}}{\tilde{d_i}} \quad \textrm{hvor} \quad i = 2,3,...,n
\end{equation}

med initialbetingelser $\tilde{d_1} = b_1$ og $\tilde{e_1} = \tilde{b_1}/\tilde{d_1}$. Vi har nå utført en forlengs substitusjon. For å finne den endelinge (diskrete) løsningen \textbf{v}, bruker vi det vi kaller \textit{baklengs substitusjon}. Vi starter med å se på det endelinge resultatet etter at alle pivotelementene har blitt radredusert til ledende enere:

\begin{equation}
    \left(\begin{array}{cccccc}
                           1 & c_1/\tilde{d_1} & 0 &0 \\
                           0 & 1 & c_2/\tilde{d_2} &0 \\
                           0 & 0 & 1 & c_3/\tilde{d_3} \\
                           0 & 0 & 0 &  1  \\
                      \end{array} \right)\left(\begin{array}{c}
                           v_1\\
                           v_2\\
                           v_3\\
                          v_4  \\
                      \end{array} \right)
  =\left(\begin{array}{c}
                           \tilde{e}_1\\
                           \tilde{e}_2\\
                        \tilde{e}_3\\
                        \tilde{e}_4\\
                      \end{array} \right)
\end{equation}

Skriver vi ut ligningene får vi at
\[v_4 = \tilde{e_4}\]
\[v_3 + (c_3/\tilde{d_3})v_4 = \tilde{e_3} \]
\[v_2 + (c_2/\tilde{d_2})v_3 = \tilde{e_2}\]
\[v_1 + (c_1/\tilde{d_1})v_2 = \tilde{e_1}\]

Den endelige algoritmen for å finne $v_i$ blir derfor på formen
\begin{equation}
v_i = \tilde{e_i}-(c_{i}/\tilde{d_i})v_{i+1} \quad \textrm{når} \quad i = n-1, n-2, ..., 1
\end{equation}

og $v_n$ = $\tilde{e_n}$ når $i$ = $n$. Figur \ref{fig:gausskode} viser hvordan disse algoritmene kan implementeres inn i C++:

\FloatBarrier
\begin{figure}[!ht]
  \begin{center}
  \includegraphics[width = 100mm]{gauss_algoritme.png}\\
  \caption{Figuren viser hvordan forlengs -og baklengs substitusjon kan implementeres i C++. }\label{fig:gausskode}
  \end{center}
\end{figure}
\FloatBarrier

Vi ser at den forlengs substitusjons-algoritmen har 6 FLOPS (floating point operations), mens baklengs substitusjon har 2 FLOPS. Vi ender altså opp med at 
\begin{equation}
N_{tridiagonal} = O(8n)
\end{equation}

\textbf{Spesialtilfelle}\\
I vårt tilfelle har elementene langs diagonalen identisk verdi. Det vil si; alle elementene $a_i$ = $c_i$ = -1, og alle elementene langs diagonalen $b_i$ = 2. Setter vi disse tallene inn i algoritmen vår, får vi for forlengs substitusjon at
\begin{equation}
\tilde{d_i} = 2 + \frac{1}{\tilde{d_i}} \quad \textrm{og} \quad \tilde{e_i} = \frac{\tilde{b_i} + \tilde{e}_{i-1}}{\tilde{d_i}}
\end{equation}

når $i$ = 2, 3, ..., $n$, mens baklengs substitusjon forenkles til
\begin{equation}
v_i = \tilde{e_i} + \frac{v_{i+1}}{\tilde{d_i}}
\end{equation}

når $i$ = n-1, n-2, ..., 1. Figur REF viser hvordan dette vil se ut i C++.

FIGUR HER

Vi ser nå at antall FLOPS har blitt redusert til 6, slik at 
\begin{equation}
N_{special} = O(6n)
\end{equation}



\subsection{LU-faktorisering}
Det finnes en annen måte å løse det lineære ligningssettet vårt på. Matrisen $A$ i ligningen $A\textbf{v}=\tilde{\textbf{b}}$ kan dekomponeres til et produkt av to matriser, nemlig $L$ og $U$, hvor $L$ er en nedre triangulær matrise og $U$ er en øvre triangulær matrise: $A=LU$

\begin{equation}
 A = \left(\begin{array}{cccccc}
                          1 & 0 & 0 &\dots    & 0 \\
                           l_{21} & 1 & 0 &\dots &0 \\
                           l_{31} & l_{32} & 1 & \dots &0 \\
                          \vdots  & \vdots   &    &\vdots   \\
                           l_{n1 }& l_{n2}  & l_{n3}  &\dots  &1 \\                         
                      \end{array} \right) 
                      \left(\begin{array}{cccccc}
                          u_{11} & u_{12} & u_{13} &\dots  & u_{1n} \\
                           0 & u_{22} & u_{23} &\dots & u_{2n} \\
                           0 & 0 & u_{33} & \dots & u_{3n} \\
                          \vdots  & \vdots   &    &\vdots   \\
                           0 & 0   & 0  &\dots  & u_{nn} \\                         
                      \end{array} \right)
\end{equation}

Vi kan nå sette inn at $A=LU$ i ligningen $A\textbf{v}=\tilde{\textbf{b}}$, slik at $L(U\textbf{v}) = \tilde{\textbf{b}}$. Vi ender opp med to ligningssett:
\begin{equation}
L\textbf{y} = \tilde{\textbf{b}} \quad \textrm{og} \quad U\textbf{v} = \textbf{y}
\end{equation}

Her finner vi først $\textbf{y}$, for så å bruke $\textbf{y}$ til å finne løsningen $\textbf{v}$.


\section{Relativ feil}
OPPGAVE d)

Vi kan finne størrelsen på den relative feilen i beregningene vi nå har gjort. Ved å implementere formelen 
\begin{equation}
\epsilon_i = log_{10} \left(\left| \frac{v_i - u_i}{u_i} \right| \right)
\end{equation} 

hvor $i = 1,...,n$. $\epsilon_i$ er da en funksjon av $log_{10}(h)$, siden verdiene $v_i$ og $u_i$ avhenger av steglengden $h$. Vi finner den maksimale relative feilen for ulike verdier av $n$, helt opp til $n=10^7$.


\section{Vedlegg}
Github-adresse: https://github.com/livewj/Project-1

%\lstinputlisting[language=Matlab]{/Users/livewj/Documents/UiO/FYS2150/Bolgeoptikk/test.m}



\bibliography{Referanser}
\begin{thebibliography}{9}

\bibitem{squires}
  $project1\_2016.pdf$
  found at the official Github-page of the course $\textit{FYS3150 - Computational Physics}$
  https://github.com/CompPhysics/ComputationalPhysics,
  03.09.2016
  
    
\end{thebibliography}

\end{document}

